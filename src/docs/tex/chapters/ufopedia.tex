%-*-texinfo-*-
%This is part of the "UFO:Alien Invasion"-Reference Manuals Tex sources.
%Copyright (C) 2006 Eric Goller.
%See the file ufo-manual_EN.tex for copying conditions.

\chapter{Ufopedia}
\section{Ufopedia}
In the following you will find a list of selected Ufopedia entries. As a big part of UFO:AIs gameplay is about research and learning about your extraterestrical enemy we do not want to spoil your fun by giving away all the secrets for free here. Also we have limited ourselfs to list things that may be helpfull to get started with the game and make reasonable decisions without starting a new game a dozen times.
\textbf{I will add pictures and stats here in the next version}

\subsection{Skills}
\subsubsection{Basic skills}
\paragraph*{Power}
 reflects a soldier's physical strength. A high physical strength is especially important for soldiers who handle heavy weapons and armor, as well as soldiers who fight in close quarters. Power directly influences the damage a soldier can do in melee combat, and how well a soldier is able to handle a weapon's recoil. Recoil decreases accuracy, so a soldier using a weapon with a lot of recoil needs strength to keep it pointing in the right direction when firing.\\
Power also affects soldier's health points (HP) and the amount of equipment a soldier can carry before he becomes encumbered. An encumbered soldier suffers a time units (TU) penalty, as well as an accuracy penalty. 
\paragraph*{Speed}
 represents how fast a soldier moves. The attribute affects mainly how much time units (TU) a soldier has. However, TU are, arguably, the most important survival characteristics of a soldier, so the speed attribute should not be underestimated. Morever the skill determines the initiative (who shoots first) when reaction fire is triggered.
\paragraph*{Accuracy}
 represents how good a soldier is at hitting a target. Accuracy is important for all soldiers who use ranged weapons, but especially so for snipers, assault weapon specialists and explosive weapon users.
\paragraph*{Mind}
 is a representation of the mental training of a soldier. The better this attribute, the less likely a soldier is to panic, and the better the soldier is at psionic warfare. Moreover, the use of utility miscellaneous equipment, as well as mines, depends on this skill. Soldiers with weak minds should not count on fast promotions.
\subsubsection{Weapon Proficiencies}
\paragraph*{Close combat}
 skill represents a soldier's proficiency with close-range weapons. A soldier with a high Close Combat skill is better at aiming a pistol at a fast moving enemy, and can also wield a blade better in combat, increasing damage. Fire modes affected by the Close Combat skill are always short range and with relatively low TU costs.\\
Examples: Combaat knife, 9mm Pistol
\paragraph*{Heavy weapons}
 skill affects how well a soldier is able to handle weapons that weigh a lot or weapons that produce much recoil. Recoil decreases accuracy tremendously, unless the soldiers is trained to counteract with the muscles of his arms. Heavy weapons include high stopping power firearms such as flame throwers and shotguns, as well as many medium and long range heavy support fire weapons. Fire modes governed by the Heavy Weapons skill always trade accuracy for damage, but differ in every other characteristics.\\
Examples: Riot Shotgun, Flamethrower
\paragraph*{Assault guns}
 skill reflects a soldier's proficiency with assault weapons. The assault skill is a combination of the ability to quickly identify friend from foe in tight combat situations, and to fire a rifle at the latter, not the former. Weapons that use the Assault Guns skill tend to be good all-round weapons, with decent damage, accuracy and rate of fire.\\
Example: Assault Rifle, SMG
\paragraph*{Sniper rifles}
 skill represents a soldier's skill to aim a weapon very accurately, provided the weapon is designed for that. Most weapons used with that skill are indeed sniper rifles, but there are exceptions --- long range, high accuracy fire mode of any weapon is bound to require Sniper Rifles skill.\\
Example: Sniper rifle
\paragraph*{Explosives}
 skill represents a soldier's ability to use grenades, and weapons with a splash damage effect (whether caused by a high-explosive ammunition or any other ammunition, even alien). A soldier with a big High-Explosives skill is better at landing the charge at the spot where he wants it: at the feet of the enemy or, if the spot is unreachable, still close enough to harm him. A soldier trained in this skill will also be more attuned to a launcher's recoil and more accustomed to its fire trajectory, increasing accuracy.\\
Example: Rocket launcher, Frag granade

\newpage

\subsection{Primary Weapons}
The alien attack on Mumbai made our situation painfully clear. Their technology is far more advanced than ours. The complete inability of Commonwealth troops to make a dent in the Mumbai offensive revealed critical weaknesses in current military training and equipment. They lost three battalions just bringing the aliens to a standstill without inflicting significant casualties. PHALANX has to overcome these odds, and to do that we need the very best human technology has to offer.

The Excalibur Program was created to find the most effective weapons on Earth by reviewing their manufacturing standards, durability, operational record, and their combat performance in the situations where we've managed to bring the aliens to battle.
%hinweis aus munition f?r infos ?

\subsubsection*{Assault Rifle}
Technical Specifications: AR-80 Assault Rifle\\
CLASSIFIED LEVEL YELLOW\\
PHALANX Extraterrestrial Response Unit\\
Technical Document, Delta Clearance\\
Filed: 20 March 2084\\
By: Cdr. Paul Navarre, R&D: Engineering Division, PHALANX, Atlantic Operations Command\\
\paragraph*{Overview}
The French LeBlanc FAA-191 (Fusil Assaut Automatique) provides the most advantageous mix of range, penetration against alien armour, and magazine size in the assault rifle class. It is a bullpup design, meaning the magazine and action of the weapon are located behind the grip to reduce overall weapon length. The 191 fires a 30-round box magazine of 4.7mm caseless ammunition, tungsten-cored steel penetrators with impressive armour-piercing capability over short and medium ranges. The round does not perform well at long range, but then assault rifles aren't meant for long-range firefights.

Though the design is far from new, first prototyped in 2057, no other rifle has fully surpassed the 191. It is accurate and quick to reload. Its rugged construction makes it dependable in combat and unlikely to break or jam. It is resistant to heat, cold, dust and humidity all at once. Ammunition and replacement parts are widely available due to the design's maturity and relative popularity among Earth's armed forces. This is a rifle you can entrust your life to.

For PHALANX use, we have given this rifle the classification AR-80.
\paragraph*{Recommended Doctrine}
This should be our go-to weapon for medium-range engagements. It can lay down an impressive amount of fire and is deadly out to far longer ranges than any pistol or SMG.

However, no one should entertain the illusion that the assault rifle is the end-all be-all of our arsenal. A balance of weapons is required for us to be able to deal with different situations; while the assault rifle remains a good weapon at close range, it can be slow to manoeuvre in tight spaces and is eclipsed by both submachine guns and shotguns at these ranges, weapons which provide a far greater point-blank punch. Any sweep of an urban building should be led by close-range weapons, while rifle-equipped soldiers either secure the perimeter or form a second assault wave to support the others.

If our rifle-equipped troops are caught beyond effective weapon range, they should make an advance from cover to cover in order to effectively bring their weapons to bear. Snipers should provide covering fire for the advancing teams or attempt to take out the enemy at range.
\paragraph*{Addenda}
While the AR-80 is a fine rifle in all respects, it was not designed to fight an alien invasion force. We should develop our own purpose-built weapons as soon as it becomes feasible to do so.

\newpage

\subsubsection*{S-1 Sniper Rifle}
Technical Specifications: S-1 Sniper Rifle
CLASSIFIED LEVEL YELLOW
PHALANX Extraterrestrial Response Unit
Technical Document, Delta Clearance
Filed: 20 March 2084
By: Cdr. Paul Navarre, R&D: Engineering Division, PHALANX, Atlantic Operations Command
[edit]
\paragraph*{Overview}
Originally an anti-materiel rifle, the Canada-built Forrester LRWS (Long Range Weapon System) has since been adopted by many countries as their principal sniper rifle. It is one of the bare handful of sniper rifles developed after 2040 that do not feature a bullpup configuration ('bullpup' meaning the magazine and action of the weapon are located behind the grip to reduce overall weapon length). It fires the massive 20mm HMG (Heavy Machine Gun) cartridge, fed by 5-round magazines which can weigh as much as one kilogramme apiece. The piston-retarded floating breech is equipped with an intricate gas dispersal system which decreases felt recoil to the level of an ordinary hunting rifle. This allows quick repeated shots on semi-automatic without any loss of accuracy.

The greatest advantages of the LRWS over other modern sniper rifles are its incredibly short barrel and light weight, the only rifle in its size class that can fire the 20mm HMG round. This is made possible by a uniquely-reinforced breech and barrel made almost entirely of tungsten and titanium alloys, able to withstand the force of the round's super-high-velocity powder. Due to its short barrel, designed for tactical urban situations, the LRWS is only effective out to approximately 1 kilometre -- barely a third of the range of a standard anti-materiel rifle -- but we estimate that PHALANX should never face a situation where this might become a problem. Its reduced accuracy is amended by a highly-advanced scope that will calculate and display intelligent bullet trajectories wherever the rifle is pointing.

The integral 'smart' bipod features a pneumatic suspension system that keeps the barrel perfectly horizontal to allow accurate fire even on broken ground. The buttstock and grip automatically mould themselves to fit any shooter. Most importantly, this rifle has racked up more alien kills in Mumbai than any other weapon deployed in the fighting.

At half the weight of other sniper rifles and twice the manoeuvrability, the LWRS offers the power and flexibility that our agents require. This weapon will not disappoint.

For PHALANX use, we have given this rifle the classification S-1.
\paragraph*{Recommended Doctrine}
Soldiers equipped with the S-1should keep their distance, fire from cover, and try to use aimed shots whenever possible. This is not an automatic weapon; a missed shot wastes time, ammunition and possibly life.

All our snipers should carry at least one backup sidearm such as the P-12 or the CRC-8 SMG, or a combat knife at the very least. Should aliens threaten a sniper at close range, he should immediately draw his sidearm. Under no circumstances should he try to use an S-1 to fend off attackers at close range. The S-1 is too slow-firing to stop an advancing alien and will quickly deplete its magazine if this is attempted. If the soldier tries to shoot his magazine dry before drawing his sidearm, it will be too late.
\paragraph*{Addenda}
Along with high-explosive rockets and grenades, this is one of the few standard-issue human weapons that are fully effective against robotic aliens.

\newpage

\subsubsection*{Flamethrower}
\paragraph*{Overview}
The FT-207 is one of the most recent additions to tactical combat. Prior versions of this weapon were not employed due to the risk of accidentally igniting surrounding objects. This model, however, greatly reduces this risk by employing FHAS, or, Focused Heat Attack System, which restricts the 'hot zone' created by the flamethrower to a very defined region. Material that is centimeters outside the 'hot zone' shows no significant temperature increase. However, the enemy or object at which you aim will surely catch fire.
\paragraph*{Battle Implications}
A weapon that shoots burning napalm from its muzzle. It has decent accuracy, but very limited range. Also useful for setting minimal, precisely chosen locations on fire, one at a time.
\begin{list}{Stats}{}
\item Damage type: fire
\item primary mode:  autofire - 30(Dmg) / 12(TU) - assault rifles
\item secondory mode: fireburst - 90(Dmg) / 20(TU) - assault rifles
\end{list}
\subsubsection*{HMPL Rocket Launcher}
CLASSIFIED LEVEL YELLOW
PHALANX Extraterrestrial Response Unit
Technical Document, Delta Clearance
Filed: 20 March 2084
By: Cdr. Paul Navarre, R&D: Engineering Division, PHALANX, Atlantic Operations Command
\paragraph*{Overview}
The South-African MPMDS (Multi-Purpose Missile Delivery System) has a classic, almost surgically clean name that completely belies its purpose. It is the heaviest infantry missile launcher on the market, able to fire anti-tank shells as easily as high-explosive rockets that turn infantry into hamburger. Originally intended as light field artillery, this weapon is fairly new; its only combat experience has been in Mumbai in the hands of Commonwealth troops. It was one of the handful of weapons that eventually turned the tide in the fighting -- and there is a good reason why.

Alien UFOs on the ground emit fantastic amounts of jamming and other EW (Electronic Warfare) activity. In fact, they emit so much of it that ordinary 'smart' missiles are rendered completely ineffective. If one type of alien EW doesn't fool the missile's relatively stupid electronic brain, another will. No amount of tinkering by Earth's military engineers has been able to fix the situation. The MPMDS, however, will remain effective against the alien invaders -- one of the few human missile launchers that can -- because its rockets carry no onboard guidance.

The rockets are 120mm monsters the size of artillery shells. They are fired out of a smoothbore barrel, fin-stabilised in flight, and have a maximum effective range of 70 metres. Surprisingly they are made up of only three parts: a rocket booster, a warhead and an impact trigger. This simplicity allows them to remain effective and reliable in the most hazardous situations. Though the rockets are not very accurate, they will devastate anything in their path, even aliens. It takes only one good hit from an MPMDS rocket to send the enemy flying.

Standard ammunition for this launcher includes: HE (High-Explosive) rockets, AA (Anti-Armour) rockets, and IC (Incendiary) rockets. New types of MPMDS ammo were being researched by Commonwealth manufacturers before the start of the war. We have all their data on file and could use it to create revolutionary new rocket types.

Along with the HPGL Grenade Launcher, this weapon will provide our troops the artillery support they need to survive and win through.

For PHALANX use, we have given this rocket launcher the classification HPML.
\paragraph*{Recommended Doctrine}
The HPML is best used in a support role, providing covering fire for our assault troops with high-explosive and/or incendiary rockets. However, care should be taken that friendly fire incidents do not occur, as this could have disastrous effects in a combat mission.

Friendly troops should be kept away from the rear of the launcher during firing; the hot exhaust gases are extremely dangerous to nearby humans. The HPML should not be fired at targets closer than 8 metres from the shooter. Violating these safety guidelines could result in serious injury or death to the shooter and other members of the team. If a target is closer than 8 metres, any shooter should immediately resort to his sidearm or a combat knife. Anything else would be suicide.
\paragraph*{Addenda}
Along with sniper rifles and grenades, this is one of the few standard-issue human weapons that are fully effective against robotic aliens.
\subsection{Secondary Weapons}
\subsubsection*{Combat Knife}
\paragraph*{Overview}
Your basic combat knife. Can be used at very short range or can be thrown.
\paragraph*{Battle Implications}
The relatively high stab damage is due to the ability to precisely aim the blade at weak spots of the opponent's body, between any armor plates. 
\subsubsection*{P-12 / 9mm Pistol}
Technical Specifications: P-12 Pistol
CLASSIFIED LEVEL YELLOW
PHALANX Extraterrestrial Response Unit
Technical Document, Delta Clearance
Filed: 20 March 2084
By: Cdr. Paul Navarre, R&D: Engineering Division, PHALANX, Atlantic Operations Command
\paragraph*{Overview}
With the return of armour to the battlefield, starting with steel helmets in World War 1 and fragmentation vests in Vietnam, pistols have had a harder and harder time keeping up. Due to their far lower muzzle velocity compared to longer-barreled and/or fully automatic weapons, they've had increasing trouble penetrating the new, ever-higher standards of human armour -- much less advanced alien composites. The very concept of the pistol in military use came under fire at one point in the 21st century, and was saved only by the advent of super-high-velocity powder.

Chief among the new generation of super-pistols is the Dolvich DV762 from Russia. It follows the design philosophy of its home country; rugged, reliable power without frills. Its design is extremely basic, and though the materials used in its construction are far stronger to cope with the new powder, the DV762's internals are no more advanced than any pistol of the late 20th century.

The DV762 does not compromise. It isn't a multi-function firearm. It's designed for only one thing: to punch through armour and kill the person inside. In order to do this, the DV fires the ancient 7.62mm Tokarev pistol round, either on semi-automatic or three-round burst mode, from a 12-round detachable box magazine. The Tokarev round is known for its excellent penetration, and it has been significantly upgraded on its return to military service. This pistol can shoot clean through the side of a modern ballistic helmet at ranges of up to 10 metres -- and then out the other side.

Unfortunately, in order to achieve penetration, the DV sacrifices stopping power. The 7.62mm round makes a very clean hole in the enemy, which is the problem; it's very reluctant to fragment or tumble, requiring a direct hit on a vital organ or major artery to incapacitate or kill an enemy.

For the purposes of the Excalibur program, we concluded that a guaranteed minor hit is better than one that may simply bounce off an alien's armour, especially as we have yet to gain a clear picture of how nasty alien armour can get. We need sidearms that we know will be effective in the crunch, and the DV762 is the best of them.

For PHALANX use, we have given this pistol the classification P-12.
\paragraph*{Recommended Doctrine}
The P-12 is primarily a backup weapon. It is a significant step up from the combat knife as a weapon of last resort, and it lets a soldier respond to new close-range threats if the primary weapon is rendered ineffective at such ranges or has run out of ammo.

Ambidextrous soldiers may even consider using two pistols at the same time, though this will negatively impact accuracy and reduce the soldier's already minimal effective range.

A single P-12 should rarely be considered as a primary weapon, as it is outclassed in this role by nearly every other weapon in our arsenal. Its advantages are the advantages of a sidearm -- small size and weight. Still, it may find a use as a primary weapon with field medics and technicians who do not have room for larger weapons.
\paragraph*{Addenda}
Despite good penetration against organics, this weapon performs very poorly against robotic targets.
\subsubsection*{CRC-8 SMG}
Technical Specifications: CRC-8 SMG
CLASSIFIED LEVEL YELLOW
PHALANX Extraterrestrial Response Unit
Technical Document, Delta Clearance
Filed: 20 March 2084
By: Cdr. Paul Navarre, R&D: Engineering Division, PHALANX, Atlantic Operations Command
\paragraph*{Overview}
From our experiences in Mumbai and other stricken cities, we've concluded that the aliens seem to concentrate their efforts on population centres, especially dense urban areas. A majority of engagements have taken place at knife-fighting range. For the purposes of the Excalibur Program, we've chosen several high-performance weapons for our Close Range Combat package.

Designed and manufactured in mainland China, the Ohm 55 SMG is one of the most frightening weapons to come out of the Second Cold War. It was first prototyped in 2035 by scientists working for the Communist Chinese government. Production models were only trickling into government units by the end of the war, but the rebel and Commonwealth troops quickly learned to respect the Ohm's ferocity.

Its rate of fire at full auto exceeds 1200 rounds per minute. It can chew through a 50-round magazine in three seconds. It fires an upgraded version of the Belgian 5.7mm armour-piercing round, a steel penetrator with aluminium core, which can tear kevlar like paper and tumbles brutally through flesh and bone. Even modern ballistic fibre cannot stop this round at anything closer than 12 metres. The Ohm 55 has dominated the field of SMGs for the past 50 years and will continue to do so for at least the next decade.

The Ohm is highly manoeuvrable with a short barrel and sleek lines, but can suffer from excessive muzzle climb on full auto due to the sheer weight of lead the weapon puts out. Autofire also tends to empty the magazine before the shooter even realises he's holding down the trigger. Still, after nearly 50 years in service around the world, this remains the Ohm 55's only known design flaw.

For PHALANX use, we have given this submachine gun the classification CRC-8.
\paragraph{Recommended Doctrine}
The CRC-8 is intended for point-blank urban firefights. It will perform very well in this role, but don't expect it to hit the broad side of a barn out to medium range. It can also function as a high-powered sidearm, but may be too bulky for most soldiers to use in this manner.

While the CRC-8 does suffer excessive muzzle climb on full auto, throwing off the aim of even experienced users, it is much more docile in its standard burst mode. Full auto should rarely be considered outside of panic situations.
\paragraph*{Addenda}
Despite good penetration against organics, this weapon performs very poorly against robotic targets. 
\subsubsection*{Riot Shootgun}
\paragraph*{Official Description}
\paragraph*{Battle Implications}
The SG-260 riot shotgun has tremendous stopping power at close range and, accordingly, considerable recoil. Its main bulk hardly fits in the holster and the double barrel, though short, slightly protrudes along the soldier leg. To master this secondary weapon, the soldier needs both hands and considerable skill with heavy weapons. 
\subsection{Misc}
\subsubsection*{Frag Granade}
\paragraph*{Official Description}
A standard grenade, something every soldier should have with him on a mission. The concept of grenades has been the same for more than a century. Prime it, throw it at your enemy, watch the enemy blow up. However, as we encounter this new alien threat, the quality and effectiveness of fragmentary grenades is increasingly questionable.
\paragraph*{Battle Implications}
The fragmentation grenade can be thrown by a soldier. Its explosion damages all units caught in its blast radius. The timer can be set between 0 and 9, where 0 means the grenade explodes immediately after being thrown, 1 means it explodes on the start of the player's next turn, etc. Timed grenades may be picked up and thrown back.
\subsection{Amor}
\subsubsection*{Kevlar Vest}
\paragraph*{Official Description}
The dense Kevlar fiber has been utilized in armor since before the turn of the millennium. Any particles hitting the vest are usually stopped due to the friction when attempting to pierce the vest and the soft fiber also absorbs the impact. It is also not flammable.
\paragraph*{Battle Implications}
A light and inexpensive armor offering considerable protection from enemy fire. Unfortunately the armor is not very durable. While Kevlar is not flammable, the other parts of the armor: stripes, pockets, etc. are easily damaged by fire or any other harsh conditions. 
