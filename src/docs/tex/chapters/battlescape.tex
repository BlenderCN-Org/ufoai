\section{Game Mechanics - Battlescape}

\subsection{Time Units (TUs)}
Every soldier has a variable amount of TUs, mainly determined by his ``speed'' attribute. Every action performed costs a different number of TUs. The amount of TUs needed for using the primary/secondary mode of an item is given in the status window after selecting one of the two, whereas other actions have fixed costs.

\subsection{Movement}
Like firing a weapon, movement also consumes time units. You can make your soldier walk to a spot using your mouse on the tactical view. You will notice that your cursor turns to a green square indicating that this place is reachable with your current amount of TUs or turn blue if it is not (this might be the case due to a lack of TUs or for geographical reasons). If the square is green it will also prompt two numbers of which the first one states the TU-cost of this movement while the second one represents your actual amount of TUs. If your soldier notices a new enemy or civilian in his line of sight while walking, the movement will be interrupted to give you the chance to change your orders.

\subsection{Line of sight}
Your soldiers can generally only shoot at what they see. After finishing an ordered movement your soldier will look in the direction of his last step, which is not very helpful in a lot of situations. You can change your soldier's facing using several different methods.  For details please refer to keybindings in the game options menu, or you can use the \keybinding{right mouse button}.

\subsection{Shooting modes}
As we have said before, many items have two different action/firing modes. While the second firing mode of a sniper rifle is an aimed shot, some assault rifles can start a long fireburst or fire one concentrated and therefore devastating single beam. Whatever weapon interests you, Ufopedia is your friend. You should also consult the Ufopedia for each individual type of ammunition, as the sort used can radically alter the characteristics of a weapon. Some ammunition will even change the skill required to use that weapon. Many entries are not available until you have researched them, so you should frequently check back with the Ufopedia.

\subsection{Close combat}
Caught with the wrong weapon armed as an alien pops up around the corner?
Your soldiers keen on some extra thrill? You want to capture a living alien for ``interrogation'' but all your research department has to offer is a stun rod, which they say \emph{may} work -- somehow? No matter what the reasons may be, there will be times you will get into close combat, or it will get to you. While the reason to be that close to an hostile alien might be quite scary, the interface for close combat is not. It works exactly like carrying a gun, but your power skill is taken into consideration when calculating the combat results. Most close combat weapons cause a large amount of damage relative to the TUs required to use them, making them a reasonable choice in confined spaces like buildings and the likes. \textbf{Hint}: Most pistols also fall under the close combat category.

\subsection{Friendly fire}
You had better make sure there are none of your soldiers in any possible line of fire when using indirect or direct firemodes -- friendly fire kills just as effectively as enemy fire.

\subsection{Reaction fire (RF)}
One of the main aspects of combat every commander needs to be able to use to his advantage is ``reaction fire'' (RF). You may recall this from the discussion of the basics of the Battlescape. 

There are two kinds of reaction fire (referred to RF-1 or RF-2 in the following). Which mode your soldier will use is indicated by one or two $\surd$s (HUD) or an ``i'' or ``*'' (altHUD). When enabled (costing a certain amount of TUs) your soldier will be able to react to new sightings during the enemy turn. When using RF-1, your soldier has one shot at any enemy that he has at least a 30\% chance to hit with no more than 5\% risk of friendly fire. Those conditions also hold for RF-2 but with this option the soldier in question fires as often as possible, using TUs available for next turn.

If your units have suffered casualties with RF activated, your soldiers may refuse your order to ``turn it off'', as they are too scared to let their guard down.  They may also take greater risks (lesser chance to hit or bigger tolerance for friendly fire) in their attempts to kill the enemy.  This is influenced by the soldier's morale, so see the relevant section for more detail.

\subsection{Damage Types}
Different weapons cause different kinds of damage, and so do different loadouts of ammunition. To reflect this, each weapon and type of ammunition is assigned a particular damage class. This is important when it comes to armour types as different armour types defend better against different damage types. Details can be found in the relevant Ufopedia entries for armour and ammunition. For instance, some armour that is almost impenetrable for plasma weapons may fail to offer any protection against fire damage.

\subsection{Stun}
In order to find out more about your alien enemy and his goal, motivation and structures you might find it useful to catch one or more them for interrogation. Once your research lab has developed the tools needed to do so, you may use them as any other weapon. After successfully finishing a mission, any stunned aliens will be brought to your base. You will first need special structures to ensure your ``guest'' will stay long enough to give you any answer at all. If you lack alien containment facilities, your stunned aliens will die instantly. If all goes well, they will open up new options in the research department.

\subsection{Morale}
Both your squaddies and the aliens have a morale stat, which plays an important role in tactical combat. In critical situations, morale can make a difference between victory and massive casualties and a failed mission.

There are a couple of influences to any character's morale and once one reaches a critical point, the result can be anything from throwing away his weapon and running away, to panic attacks -- including shooting at allied forces.

A character's morale will drop slightly when he witnesses a civilian being killed. If the same happens to a fellow squaddie, his morale drops still more.  On the other hand, if an alien dies nearby, his morale will increase. All that relative to the soldiers morale values.
