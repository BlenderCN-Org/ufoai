\section{Game-mechanics (Battlescape)}

\subsection{Time-units (TUs)}
As mentioned before every soldier has a certain amount of time units (TUs) which are mainly determined by his ``speed'' attribute. Every action done by him costs a varying amount of TUs this holds for firing or reloading a weapon as well as walking or re-equipping him in the inventory. The amount of TUs needed for using the primary/secondary mode is given in the status window after selecting one of the two.

\subsection{Movement}
Like firing a weapon, movement also consumes time units. You can make your soldier walk to a spot using your mouse on the tactical view. You will notice that your cursor turns to a green square indicating that this place is reachable with your current amount of TUs or turn blue if it is not (this might be the case due to a lack of TUs or for geographical reasons). If the square is green it will also prompt two numbers of which the first one states the TU-cost of this movement while the second one represents your actual amount of TUs. In case your soldier notices a new enemy or civilian in his line of sight while walking the movement will be interrupted, giving you the chance to adjust your orders according to this new situation.

\subsection{Line of sight}
For obvious reasons you soldiers, in general, can only shoot at what they see. After finishing an ordered movement your soldier will look in the direction of his last step, which is not very helpful in a lot of situations. To solve this you might make use of the possibility to change your soldiers viewing direction. This can be done in different ways, e. g. \keybinding{Right-mouse-button} / \keybinding{CTRL}. For details please refer to your keybindings in the game options menu.

\subsection{Shooting-modes}
As we have said before most items, weapons in particular, do have two different action/firing-modes. While the second firing-mode of a sniper rifle is an aimed shot, some assault rifles can start a long fireburst or fire one concentrated and by that devastating single beam. Whatever weapon raises you interest, Ufopedia is your friend. If you look up a certain weapon like that you might be confused, the only information that can be found here is its name and if its a two-handed one or not. What seems rather wired on first sight has a simple reason. As some weapons can be equipped with a wide range of different kinds of ammunition their use and stats also heavily depend on the ammunition loaded. So once you look up the ammunition you want to use you will find all the data and statistics you are looking for - given you have done the required research. Doing so you will find that different firing-modes not only differ by TU needed and damage done but also by weapon skills needed.

\subsection{Close combat}
An alien is popping up just around the corner and not enough TUs left to fire this Plasma-Blaster in secondary mode while primary-mode offers only indirect fire? Your soldiers being keen on some extra thrill? You want to capture an living alien for ``interrogation'' but all your research department has to offer is a stun rod of which they say it might work --- somehow\ldots? No matter what the reasons may be, there will be a time you will get into close-combat, or it will get to you. While the reason to be that close to an hostile alien might be quite scary, lucky enough the way to use the interface in such a situation is not at all. Overall it works exactly like caring a gun besides the fact that your power skill is taken into consideration when calculating the combat results. Also most close combat weapons (that includes pistols as well) do have a far more devastating impact on their target compared for their needed TUs making them a reasonable choice in small and narrow environments like buildings and the likes. Hint: Most pistols also fall under the close combat category which makes them a useful alternative.

\subsection{Friendly fire}
You better make sure there is no one of your soldiers in any possible line of fire when using RF or normal firemodes - friendly fire is rather strict right now.

\subsection{Reaction fire}
One of the main aspects every experienced commander needs to be able to use for his advantage is what is called ``reaction fire'' (RF). When discussing the basics of battlescape we already mentioned its button but spared to explain the corresponding concept. To make things even more complicated there are two kinds of reaction fire (referred to RF-1 or RF-2 in the following). The RF-mode that is activated is indicated by one or two $\surd$ s (HUD) or an ``i'' or ``*'' (altHUD). When enabled (costing a certain amount of TUs) your soldier will be able to react on new situations and sightings after you already ended your turn. For doing so in case of RF-1 he has one shot on any enemy that he has at least a 30\% chance to hit with no more than 5\% risk of friendly fire. Those conditions also hold for RF-2 but with this option the soldier in question fires as often as possible while he has all the TUs of the upcoming round (without costs for RF) at his disposal to fire his weapon in order to deal with this more or less surprising situation. Especially after having suffered heavy penetration by enemy fire with reaction-fire activated, your soldiers will refuse your order to ``turn it off'' as they are too scared to let their guard down or will take greater risks (lesser chance to hit or bigger tolerance to friendly fire) in their approach to kill the enemy. For details about this and other effects of a bad moral please refer to the according section of the manual.

\subsection{Damagetypes}
Obviously different weapons cause different kinds of damage. To reflect this fact each weapon is assigned a certain damage-class. This gets important when it comes to armor types as different armor types suit different damage types. Details can be found in the according armor and ammunition Ufopedia entries. This way it might be possible that some armor that is almost impenetrable for plasma damage fails to offer any protection against weapons that inflict fire damage.

\subsection{Stun}
In order to find out more about your alien enemy and his goal, motivation and structures you might find it useful to catch one or more them for direct interrogation. Once your research lab developed the tools needed to do so you might use them as any other weapon of this kind (e.g. grenades, close-combat, etc.). After successfully finishing a mission those stunned alien will be brought to your base. Please be aware that you will need special structures to make sure your ``guest'' will stay long enough to give you any answer at all. If you lack those facilities your stunned aliens will die instantly once you've reached your base. In case everything is prepared to make your stunned aliens feel at home they will open up new options in the research department.

\subsection{Morale}
Your squads, but also your enemies, morale plays an important role in tactical combats. Especially in critical situations that tend to bring the decision on win or lose.\\
There are a couple of influences to any characters moral and once one reaches a critical point the result can be anything from throwing away your weapon and running away to panic attacks including shooting at allied forces.\\
A characters moral is going to drop slightly when is witnesses a civilian being killed. If the same happens to a squadmember his moral drop far more remarkable and if an alien dies nearby on the other hand moral is going to increase. All that relative to the soldiers moral values.
