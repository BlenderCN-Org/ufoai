\section{Your base}
You will use your bases in order to accomplish a wide range of tasks, ranging from the researching and production of new equipment to gathering background information on the invaders. You can change the name of your bases by clicking on the pen icon right next to its name on the base screen. You can also cycle through all your current bases using the arrow icons. The following subsections describe the basics of base management.

\subsection{Buildings}
This is where you order the construction of additional facilities for your base.  Building laboratories will allow you to increase your research cap, hospitals will allow you to hasten your soldiers' healing, and workshops allow you to produce more equipment.  Before you place a new building, make sure you have read its Ufopedia entry. There you can find out what the building is used for, or if it requires additional buildings to be constructed. Keep in mind that at least a power plant and command centre are needed for most other buildings to be useful. Another important aspect when expanding your base is building time -- buildings vary quite a bit in this regard.  Don't forget, new bases can be built in other locations, so you don't need to place all facilities in one location.  Since space on individual bases is limited, you will need to carefully select what to build where.

% Screenshot?
\subsection{Aircraft}
This menu brings up a screen where you manage the aircraft at that particular base. This includes not only equipping your vessels, but also buying new ones or transferring them to another base. You can cycle through all your aircraft using the left and right arrow icons. From here, you can also call a ship back to base or launch it, although you're more likely to want to do this from the \emph{Geoscape} screen.

Probably the most important sub-menu here is \emph{Equip Aircraft}.  This brings up a screen which allows you to choose which soldiers to assign to your selected aircraft. A standard dropship has room for 8 soldiers, and you will generally want to use all of them (unless you like a challenge, of course). In order to choose the best soldiers for an upcoming mission, you are provided with a picture of your selected character and his statistics. A simple click on the `X' or$\surd$ assigns or removes the selected soldier from the current ship. You may also rename your fighters using the ``edit'' button in the upper right, just next to current soldier's name. Also please notice that while you can assign one soldier to an interceptor ship, this currently will do you no good.

Once you have decided who to take to the battlefield, you must confirm your selection using the button in the very bottom right corner.  At this point an inventory screen will come up. Provided the ship in question has yet to leave, you can re-do your troop selection as often as you want.

At the inventory screen, you can equip your soldiers for their upcoming missions. In the upper left you see all soldiers assigned to the current aircraft. On the opposite side of the screen, you see the soldier with his inventory. The amount of space an item requires is represented by the number of squares it covers. The biggest part of the screen is used by your base's item stock. You choose which of the four categories of equipment (Primary, Secondary, Miscellaneous, Armour) to display using the appropriate button, then drag and drop items from the base stock into your soldier's inventory.  Weapons shown with a red background lack the required ammo and aren't useable. You may equip them anyway, but unless you get the required ammunition from somewhere else they won't be of any use. Every soldier has different weapon skills, and the lower left hand side of the display will show these to you. Some weapons utilise different weapon proficiencies, depending on the chosen firemode. Alternatively to the soldier's stats window you can change this to an object details view which presents the basic stats of an item. For details on damage and firemodes of a weapon you need to view the details of the relevant clip / ammunition, as some weapons can be equipped with different types of ammo. Clicking on the arrow symbol in the bottom right corner confirms your selections and gets you back to the aircraft screen.

\subsection{Buy / Sell Equipment}
Here you can get new equipment from the global market or get rid of any item for which you have no further use. Items not carried by your soldiers at the end of a mission are sold automatically, the details of which will be displayed on the summary screen. If you want to use the items captured you can simply buy them back here. Currently there is no difference between the purchase and sale price of individual items, so you won't lose money in these transactions.  However, future versions of the game will include a better economy model and so this will likely change.  The amount of any kind of items available may change in the course of the game with your world reputation.  Again, all items are again sorted into four categories.

\subsection{Transfer}
Here you can transfer your equipment between different bases.
%moretocome

\subsection{Research}
Research is a critical factor in your attempts to defend earth against the alien threat, so it is essential to keep your R \and D department busy. It is important to research all available aspects, although the temptation will be to research weapons first.  On the \emph{Research} screen, the left part gives all possible research options.   The right part shows details on the selected subject. In order to discover new research options it's usually necessary to capture examples of the appropriate item, an alien body, or even a live alien.  Sometimes a simple prototype of some alien tech is not enough to get your research started. In such cases the research option is given in grey letters as it requires further research on some other more basic field beforehand. The concrete dependencies for each technology are given in its details shown on the right side of the screen.

To assign scientists to a research project just use the left and right arrows next to the technology in question. The left arrow will add scientists to the research project, while the right one will re-assign them to the pool.  The actual progress is given in the left window. Hint: while it is possible to work on several technologies at the same time it is usually better to focus on one project at a time.

\subsection{Production}
Here you can built equipment that is not available on global market, or that is a result of the efforts of your research department. To order an item to be built, select it on the left part of the screen and adjust the number to be built using the arrows under its image. The production cost of one item is initially taken from your cash if you are building more than one. For example, while 3 assault rifles cost 63000, you need only 21000 to start production.
% What happens when you run out of cash, production just ceases until you get more money?

\subsection{Hire employees}
Using this screen you can add more personnel to your organisation. While especially in the beginning people do not trust your ability to counter the aliens, they might be more enthusiastic (and therefore willing to work for you) as you proceed. On the left side you find all members of one group (soldiers, medics, workers, scientists) listed. Clicking on the `X' or $\surd$ hires or fires them. You can hire and fire  them as often as you want, they will never get angry at you. But please be aware that personnel you hire in one base won't be accessible from another base. So if you want to fire someone make sure you are in the corresponding base. Also you should keep in mind that the amount of personnel that can work in your base might be limited by the base's housing or working facilities.
% necessary?
% AFAIK this is not the case right now. also it doesn't seem possible to hire more than 19 persons of one group for the simple reason that % there is no way to scroll down the list.
